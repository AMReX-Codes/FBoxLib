\noindent \FBoxLib\ is a software library containing all the functionality 
to write massively parallel, 
block-structured adaptive mesh refinement (AMR) applications in two and three dimensions.
\FBoxLib\ is developed at the Center for Computational Sciences and Engineering (CCSE) at 
Lawrence Berkeley National Laboratory and is freely available
at {\tt https://github.com/AMReX-Codes/FBoxLib}.
The most current version of this User's Guide
can be found in the \FBoxLib\ git repository at {\tt FBoxLib/Docs}.  Any questions,
comments, suggestions, etc., regarding this User's Guide should be directed
to Andy Nonaka of CCSE ({\tt AJNonaka@lbl.gov}).  Further information 
about \FBoxLib\ can be found by contacting Ann Almgren
({\tt ASAlmgren@lbl.gov}) and Weiqun Zhang ({\tt WeiqunZhang@lbl.gov})
or by visiting our website, {\tt ccse.lbl.gov}.\\ 

\noindent This is the Fortran version of \BoxLib\.  \FBoxLib\ is deprecated and has
been superseded by AMReX ({\tt https://github.com/AMReX-Codes/amrex}).\\ 

\noindent \BoxLib\ Version 2014-02-28, Copyright (c) 2014, The Regents of the
University of California, through Lawrence Berkeley National
Laboratory (subject to receipt of any required approvals from the U.S.
Dept. of Energy).  All rights reserved.\\ 

\noindent If you have questions about your rights to use or distribute this
software, please contact Berkeley Lab's Technology Transfer Department
at  TTD@lbl.gov.\\ 

\noindent NOTICE.  This software is owned by the U.S. Department of Energy.  As
such, the U.S. Government has been granted for itself and others
acting on its behalf a paid-up, nonexclusive, irrevocable, worldwide
license in the Software to reproduce, prepare derivative works, and
perform publicly and display publicly.  Beginning five (5) years after
the date permission to assert copyright is obtained from the U.S.
Department of Energy, and subject to any subsequent five (5) year
renewals, the U.S. Government is granted for itself and others acting
on its behalf a paid-up, nonexclusive, irrevocable, worldwide license
in the Software to reproduce, prepare derivative works, distribute
copies to the public, perform publicly and display publicly, and to
permit others to do so.
